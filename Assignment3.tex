\documentclass[letterpaper,10pt]{article}

\usepackage{geometry}
\usepackage{hyperref}
\geometry{textheight=8.5in, textwidth=6in}

\newcommand{\toc}{\tableofcontents}

\def\name{Helena A. Bales}
\title{CS 444 Project 3: Encrypted Block Device}
\author{Helena A. Bales, Kathryn Maule, and Alexander Scott Wilson }

\parindent = 0.0 in
\parskip = 0.1 in

\begin{document}
\maketitle

\begin{abstract}
This assignment will involve the modification of the Linux Yocto kernel. 
The kernel will be modified to add a custom encrypted block device, which we will design and implement. 
The solution will use the Linux Kernel's Crypto API to add encryption when reading and writing data. 
The solution will comprise of a RAM Disk device driver which allocates blocks of memory. 
The solution will first be implemented in the kernel directly for ease of testing, and will subsequently
 be moved to a module with module parameters used for the encryption and decryption of the data. 
This document will provide documentation and design details for the encrypted block device driver.
\end{abstract}

\clearpage

\tableofcontents

\clearpage

\section{Program Design}
We will be implementing a RAM Disk Device Driver in the Linux Yocto Kernel. 
This will be performed in several steps for ease of implementation and testing. 
The first step will be to directly implement the Encrypted Block device in the Linux kernel. 
Once that has been tested and proven to work, we will move the code into a module. 
Then we will test the module's behavior, including parameters. 
We will be implementing the module in this way because modules are significantly harder to test than a 
direct implementation, so adding the extra step of direct implementation before extraction should make 
the testing process easier.

The RAM Device driver will allocate a chunk of memory and present it to be used as a block device. 
The driver will use the Linux Kernel's Crypto API to encrypt and decrypt data when it is written and 
read. The key for the encryption will be set as a module parameter, and the text should only be 
decrypted correctly when the key is correct. The driver must also handle the case where the block is not
 big enough for the amount of data. In this case, a second block must be created to be used for the 
portion of the data that does not fit in the first block.

\section{Version Control Logs}
\begin{tabular}{l l l}\textbf{Detail} & \textbf{Author} & \textbf{Description}\\\hline
\href{https://github.com/balesh2/CS444-3/commit/6379fd575b665ad1a2ee5b224fd9f2011d5eaa09}{6379fd5} & balesh2 & starts write up\\\hline\end{tabular}


\section{Work Log}
\subsection{November 22nd, 2016 - 10:20 am}
Started write up, including documentation and design of encrypted block device driver. 
Added script to generate version control log table from git history.

\subsection{November 22nd, 2016 - 10:30 am}
Added Makefile. Updated Makefile to include the autogeneration of the git commit log when running make. 
Added a line in the makefile to remove the changelog.tex file with make clean so that a new one gets 
generated when make is run.

\subsection{November 22nd, 2016 - 10:50 am}
Added sections in documentation for purpose of assignment and testing. Started content for testing 
section and finished content for purpose of assignment section.

\subsection{November 22nd, 2016 - 11:40 am}
Added content to program design section.

\section{Purpose of Assignment}
The purpose of this assignment is to understand encrypted block devices through the implementation of 
our own RAM Disk device driver in the Linux Yocto Kernel. 
Our solution will take the form of a kernel module, for the purpose of understanding how to develop a 
kernel module. 
The driver will encrypt and decrypt the data as it is read and written, and will handle the case where 
the block is not big enough for the amount of data. 
Through implementing this driver, we will learn how drivers are written, how block devices work, and 
how encryption is implemented. 
We will additionally learn how to test a kernel module, encryption, and block devices.

\section{Testing Methods}
Since testing kernel modules is difficult, we will begin by directly implementing the RAM Disk device 
driver in the kernel, without encryption, then adding encryption, then extracting to a module. 
This will allow us to easily test the basic fuctionality first, then once that has been proven to work 
correctly, extract to a module and test the module functionality.

In order to test the functionality of the driver when it is directly included in the kernel without 
encryption we will give it different sets of data to write in then read. If the returned data is the 
same as the written data, then we can say that it behaves correctly. The standard case would be a string
 of length shorther than can be stored in the block. An edge case would be passing in a string of length
 zero, and one that is longer than the block.

In order to test that the encryption is behaving correctly, we will use the same test cases as for the 
unencrypted case, but with the added layer of testing the key. 
For each of the cases described in the previous paragraph, we will give the correct key and an incorrect
 key and check that they data is only decrypted when the correct key is presented.

Finally, we will test the module. TODO

\section{Learning Outcomes}

\end{document}
